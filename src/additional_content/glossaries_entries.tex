%% Regular glossary entries
\newglossaryentry{database}{
  name={base de données},
  description={Collection organisée de données structurées},
  plural={bases de données},
  first={base de données (BDD)},
  sort={base de donnees}
}

\newglossaryentry{algorithm}{
  name={algorithme},
  description={Ensemble d'instructions pour résoudre un problème}
  % Removed the see reference to non-existent entry
}

\newglossaryentry{complexity}{
  name={complexité},
  description={Mesure des ressources nécessaires pour exécuter un algorithme}
}

%% Acronyms
\newacronym{cpu}{CPU}{Central Processing Unit}
\newacronym[longplural={Unités Centrales de Traitement}]{uc}{UC}{Unité Centrale}
\newacronym[description={Interface permettant la communication entre les systèmes}]{api}{API}{Application Programming Interface}

%% Symbols - convert from glsxtrnewsymbol to newglossaryentry
\newglossaryentry{omega}{
  type=symbols,
  name={\ensuremath{\omega}},
  description={Vitesse angulaire},
  sort={omega}
}

\newglossaryentry{ohm}{
  type=symbols,
  name={\ensuremath{\Omega}},
  description={Résistance électrique},
  sort={ohm}
}

%% Abbreviations (with glossaries-extra)
\newabbreviation{etc}{etc.}{et cetera}
\newabbreviation[longplural={Numéros d'identification}]{id}{ID}{Identifiant}

%% Categories
\glssetcategoryattribute{math}{glossname}{Termes mathématiques}
\newglossaryentry{integral}{
  category={math},
  name={\ensuremath{\int}},
  description={Opération mathématique pour calculer l'aire sous une courbe}
}

%% Entry for the symbols glossary
\newglossaryentry{pi}{
  type=symbols,  % Add this line to specify the glossary type
  name={\ensuremath{\pi}},
  description={Rapport entre la circonférence d'un cercle et son diamètre},
  sort={pi}
}